\chapter{绪论}
\chaptermark{绪论}
\section{早期计算设备(1940年前)}
\subsection{机械计算器的诞生}
算盘作为最早的计算工具,其历史可以追溯到公元前2400年。1642年法国数学家帕斯卡发明了第一台机械计算器Pascaline。

\subsection{分析机的构想}
1837年英国数学家巴贝奇提出分析机概念,具备现代计算机的五大基本组成部分,被认为是通用计算机的理论雏形。

\section{电子计算机时代(1940-1970)}
\subsection{ENIAC的诞生}
1946年在美国宾夕法尼亚大学研制成功,使用17468个电子管,重达30吨,标志着第一台通用电子计算机的诞生。

\subsection{晶体管计算机}
1954年贝尔实验室研制出第一台全晶体管计算机TRADIC,计算机进入第二代发展阶段,体积和功耗大幅降低。

\subsection{集成电路时代}
1958年德州仪器的杰克·基尔比发明集成电路,1964年IBM推出System/360系列,确立了计算机体系结构的标准化。

\section{个人计算机革命(1970-2000)}
\subsection{微处理器的突破}
1971年Intel 4004微处理器问世,单个芯片集成2300个晶体管,开创了微处理器时代。

\subsection{个人计算机普及}
1975年Altair 8800问世,1981年IBM PC发布,1984年苹果Macintosh引入图形界面,计算机开始进入家庭和办公室。

\section{现代计算机发展(2000-至今)}
\subsection{移动计算时代}
2007年iPhone问世,标志着移动互联网时代的开始。ARM架构处理器在移动设备领域占据主导地位。

\subsection{云计算与量子计算}
AWS等云服务提供商兴起,量子计算机实现"量子优越性",计算机发展进入新的范式转移阶段。

\section{插入公式}
\begin{equation}
a + b = c
\end{equation}

引用参考文献\cite{NWPUThesisLaTeXTemplate},\gls{massEnergyFunc}及缩略\gls{npu}。